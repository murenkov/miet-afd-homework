\chapter{Информационный поиск по теме <<Кольцевой развязанный делитель>>}

Кольцевой развязанный делитель (Wilkinson power divider) --- это класс делителей мощности СВЧ-диапазона, одним из главных преимуществом которых является хорошая развязка по выходам при согласованной нагрузке на всех портах~\cite{pozar2011microwave}.
Кольцевой развязанный делитель может использоваться и в качестве сумматора в связи с тем, что выполнен с использованием только пассивных компонентов.

Как правило, кольцевые развязанные делители являются гибридными, т.е. мощность делится пополам.
В таком случае делитель можно задать следующей матрицей рассеяния (на моделируемой частоте):
\begin{equation}
    [S] =
    -\frac{j}{\sqrt 2}
    \begin{pmatrix}
        0 & 1 & 1 \\
        1 & 0 & 0 \\
        1 & 0 & 0
    \end{pmatrix}
\end{equation}

\section{История}

Впервые кольцевой развязанный делитель был представлен Эрнестом Вилкинсоном в 1960~г.~\cite{wilkinson1960}.
Описанный Вилкинсоном делитель выгодно отличался от существовавших ранее, потому что не только мог обеспечить равенство фаз и амплитуд на выходах независимо от частоты, но также имел отличную развзязку и согласование портов.
Строился делитель на четвертьволновых резонансных отрезках.

Первые кольцевые развязанные делители позволяли делить мощность только в равных пропорциях.
В 2001~г. был описан вариант делителя, позволяющий делить мощность в пропорции 4:1~\cite{lim2001wilkinson}.
Это достигается засчёт использования разветвляющихся отрезков различной ширины.

В 2002~г. Максимилианом Скарделлетти был описан способ миниатюризации кольцевого развязанного делителя засчёт использования отрезков меньшей длины, такой как $\cfrac{\lambda}{5}$ и $\cfrac{\lambda}{12}$~\cite{scardelletti2002wilkinson}.
Такое решение позволяет уменьшить размер схемы на 74~\%, но это сопровождается небольшим увеличением вносимых потерь (Insertion Loss, IL) и уменьшением ширины рабочей полосы частот.

Вполне очевидно, что настроенный на некоторую частоту $f$ делитель будет обладать похожими характеристиками и на всех кратных ей частотах.
Что делать, если этот вариант не устраивает?
Как вариант, можно отфильтровать всё лишнее, но есть и другой способ.
Так в 2003~г. был описан вариант делителя, позволяющий подавить n-ную гармонику~\cite{yi2003wilkinson}.
Для этого в середину четвертьволновых отрезков ставятся разомкнутые шлейфы длиной $\frac{\lambda}{4n}$.
Другим вариантом подавления гармоник является создание спиральных дефектных заземлённых структур (Spiral Defected Ground Structure, Spiral DGS)~\cite{woo2005wilkinson}.
Такой способ позволяет подавить более, чем одну гармонику.

А что если надо решить обратную задачу?

Есть способы обеспечить работу кольцевого развязанного делителя на нескольких частотах.
Достигается это различными методами~\cite{cheng2007wilkinson, chongcheawchamnan2006wilkinson, oraizi2006wilkinson, park2008wilkinson, wu2006wilkinson, wu2009wilkinson, wu2011wilkinson}.
Во-первых, можно поставить дополнительные участки, а в параллель резистору включить индуктивность и ёмкость~\cite{wu2006wilkinson}.
Во-вторых, каскадным соединением нескольких делителей, сопротивление резонансных участков которых расчитывается по определённым формулам~\cite{chongcheawchamnan2006wilkinson}.
В-третьих, можно воспользоваться методом, чем-то похожим на описанный ранее в разделе про подавление n-ных гармоник~\cite{park2008wilkinson}.
Как решить обратную задачу похожим способом?
Использовать разомкнутые шлейфы с другими параметрами!
Правда есть доля скепсиса в отошеинии предложенного метода, потому что он обеспечивает работу на кратной гармонике, что вроде как и так обеспечивается устройством классического кольцевого развязанного делителя.

В настоящее время развитие идеи, которой уже больше 80~лет, не прекращается.
Так как в последние годы наблюдается бурный прогресс машинном обучения, основанном в том числе и на применении нейронных сетей, появляются возможности применения этих технологий в проектировании СВЧ-элементов.
В 2020~г. было описано применение нейронной сети в проектировании кольцевого развязанного делителя, обеспечивающего подавление нескольких кратких гармоник сразу~\cite{jamshidi2020wilkinson}.

\printbibliography\
